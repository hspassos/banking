\documentclass[
	% -- opções da classe memoir --
	12pt,				% tamanho da fonte
	openright,			% capítulos começam em pág ímpar (insere página vazia caso preciso)
	twoside,			% para impressão em recto e verso. Oposto a oneside
	a4paper,			% tamanho do papel. 
	% -- opções da classe abntex2 --
	%chapter=TITLE,		% títulos de capítulos convertidos em letras maiúsculas
	%section=TITLE,		% títulos de seções convertidos em letras maiúsculas
	%subsection=TITLE,	% títulos de subseções convertidos em letras maiúsculas
	%subsubsection=TITLE,% títulos de subsubseções convertidos em letras maiúsculas
	% -- opções do pacote babel --
	brazil,			% idioma adicional para hifenização
	french,				% idioma adicional para hifenização
	spanish,			% idioma adicional para hifenização
	english				% o último idioma é o principal do documento
	]{abntex2}


\usepackage{lmodern}
\usepackage[T1]{fontenc}		% Selecao de codigos de fonte.
\usepackage[utf8]{inputenc}		% Codificacao do documento (conversão automática dos acentos)
\usepackage{indentfirst}		% Indenta o primeiro parágrafo de cada seção.
\usepackage{color}				% Controle das cores
\usepackage{graphicx}			% Inclusão de gráficos
\usepackage{microtype} 			% para melhorias de justificação

\usepackage{lipsum}				% para geração de dummy text
% ---
\usepackage{amsmath, amsthm, amssymb}


\usepackage[alf]{abntex2cite}	% Citações padrão ABNT

\usepackage{comment}

% ---
% Informações de dados para CAPA e FOLHA DE ROSTO
% ---
\titulo{Banking competition in Brazil}
\autor{Henrique de Souza Passos}
\local{Brasil}
\data{2025}
\orientador{Victor Gomes}
\instituicao{%
  Universidade de Brasília -- UnB
  \par
  Faculdade de Economia, Administração, Contabilidade e Gestão de Políticas Públicas
  \par
  Programa de Pós-Graduação em Economia}
\tipotrabalho{Dissertação (Mestrado)}
% O preambulo deve conter o tipo do trabalho, o objetivo, 
% o nome da instituição e a área de concentração 
\preambulo{Dissertação apresentada como requisito par-
cial para a obtenção do título de Mestre em
Economia.}
% ---


% ---
% Configurações de aparência do PDF final

% alterando o aspecto da cor azul
\definecolor{blue}{RGB}{41,5,195}

% informações do PDF
\makeatletter
\hypersetup{
     	%pagebackref=true,
		pdftitle={\@title}, 
		pdfauthor={\@author},
    	pdfsubject={\imprimirpreambulo},
	    pdfcreator={LaTeX with abnTeX2},
		pdfkeywords={abnt}{latex}{abntex}{abntex2}{trabalho acadêmico}, 
		colorlinks=true,       		% false: boxed links; true: colored links
    	linkcolor=blue,          	% color of internal links
    	citecolor=blue,        		% color of links to bibliography
    	filecolor=magenta,      		% color of file links
		urlcolor=blue,
		bookmarksdepth=4
}
\makeatother
% --- 

% ---
% Posiciona figuras e tabelas no topo da página quando adicionadas sozinhas
% em um página em branco. Ver https://github.com/abntex/abntex2/issues/170
\makeatletter
\setlength{\@fptop}{5pt} % Set distance from top of page to first float
\makeatother
% ---

% ---
% Possibilita criação de Quadros e Lista de quadros.
% Ver https://github.com/abntex/abntex2/issues/176
%
\newcommand{\quadroname}{Quadro}
\newcommand{\listofquadrosname}{Lista de quadros}

\newfloat[chapter]{quadro}{loq}{\quadroname}
\newlistof{listofquadros}{loq}{\listofquadrosname}
\newlistentry{quadro}{loq}{0}

% configurações para atender às regras da ABNT
\setfloatadjustment{quadro}{\centering}
\counterwithout{quadro}{chapter}
\renewcommand{\cftquadroname}{\quadroname\space} 
\renewcommand*{\cftquadroaftersnum}{\hfill--\hfill}

\setfloatlocations{quadro}{hbtp} % Ver https://github.com/abntex/abntex2/issues/176
% ---

% --- 
% Espaçamentos entre linhas e parágrafos 
% --- 

% O tamanho do parágrafo é dado por:
\setlength{\parindent}{1.3cm}

% Controle do espaçamento entre um parágrafo e outro:
\setlength{\parskip}{0.2cm}  % tente também \onelineskip


\makeindex
\begin{document}

\selectlanguage{english}
%\selectlanguage{brazil}

% Retira espaço extra obsoleto entre as frases.
\frenchspacing 
\imprimircapa
\imprimirfolhaderosto*

% ---
% Inserir a ficha bibliografica
% ---

% Isto é um exemplo de Ficha Catalográfica, ou ``Dados internacionais de
% catalogação-na-publicação''. Você pode utilizar este modelo como referência. 
% Porém, provavelmente a biblioteca da sua universidade lhe fornecerá um PDF
% com a ficha catalográfica definitiva após a defesa do trabalho. Quando estiver
% com o documento, salve-o como PDF no diretório do seu projeto e substitua todo
% o conteúdo de implementação deste arquivo pelo comando abaixo:
%
% \begin{fichacatalografica}
%     \includepdf{fig_ficha_catalografica.pdf}
% \end{fichacatalografica}

\begin{fichacatalografica}
	\sffamily
	\vspace*{\fill}					% Posição vertical
	\begin{center}					% Minipage Centralizado
	\fbox{\begin{minipage}[c][8cm]{15.5cm}		% Largura
	\small
	\imprimirautor
	%Sobrenome, Nome do autor
	
	\hspace{0.5cm} \imprimirtitulo  / \imprimirautor. --
	\imprimirlocal, \imprimirdata-
	
	\hspace{0.5cm} \thelastpage p. : il. (algumas color.) ; 30 cm.\\
	
	\hspace{0.5cm} \imprimirorientadorRotulo~\imprimirorientador\\
	
	\hspace{0.5cm}
	\parbox[t]{\textwidth}{\imprimirtipotrabalho~--~\imprimirinstituicao,
	\imprimirdata.}\\
	
	\hspace{0.5cm}
		1. Banking.
		2. Competition.
		2. Credit Spread.
		I. Victor Gomes.
		II. Universidade de Brasília.
		III. Programa de Pós-Graduação em Economia.
		IV. Título 			
	\end{minipage}}
	\end{center}
\end{fichacatalografica}
% ---

% Isto é um exemplo de Folha de aprovação, elemento obrigatório da NBR
% 14724/2011 (seção 4.2.1.3). Você pode utilizar este modelo até a aprovação
% do trabalho. Após isso, substitua todo o conteúdo deste arquivo por uma
% imagem da página assinada pela banca com o comando abaixo:
%
% \begin{folhadeaprovacao}
% \includepdf{folhadeaprovacao_final.pdf}
% \end{folhadeaprovacao}
\begin{comment}
\begin{folhadeaprovacao}

  \begin{center}
    {\ABNTEXchapterfont\large\imprimirautor}

    \vspace*{\fill}\vspace*{\fill}
    \begin{center}
      \ABNTEXchapterfont\bfseries\Large\imprimirtitulo
    \end{center}
    \vspace*{\fill}
    
    \hspace{.45\textwidth}
    \begin{minipage}{.5\textwidth}
        \imprimirpreambulo
    \end{minipage}%
    \vspace*{\fill}
   \end{center}
        
   Trabalho aprovado. \imprimirlocal, 24 de novembro de 2012:

   \assinatura{\textbf{\imprimirorientador} \\ Orientador} 
   \assinatura{\textbf{Professor} \\ Convidado 1}
   \assinatura{\textbf{Professor} \\ Convidado 2}
   %\assinatura{\textbf{Professor} \\ Convidado 3}
   %\assinatura{\textbf{Professor} \\ Convidado 4}
      
   \begin{center}
    \vspace*{0.5cm}
    {\large\imprimirlocal}
    \par
    {\large\imprimirdata}
    \vspace*{1cm}
  \end{center}
  
\end{folhadeaprovacao}
\end{comment}
% ---
\begin{dedicatoria}
   \vspace*{\fill}
   \centering
   \noindent
   \textit{} \vspace*{\fill}
\end{dedicatoria}
% ---
\begin{agradecimentos}

\end{agradecimentos}
% ---
\begin{epigrafe}
    \vspace*{\fill}
	\begin{flushright}
		\textit{}
	\end{flushright}
\end{epigrafe}
% ---
% resumo em português
\setlength{\absparsep}{18pt} % ajusta o espaçamento dos parágrafos do resumo
\begin{resumo}


 \textbf{Palavras-chave}: latex. abntex. editoração de texto.
\end{resumo}

% resumo em inglês
\begin{resumo}[Abstract]
 \begin{otherlanguage*}{english}
   This is the english abstract.

   \vspace{\onelineskip}
 
   \noindent 
   \textbf{Keywords}: latex. abntex. text editoration.
 \end{otherlanguage*}
\end{resumo}
% ---
% inserir lista de ilustrações
% ---
\pdfbookmark[0]{\listfigurename}{lof}
\listoffigures*
\cleardoublepage
% ---
% inserir lista de tabelas
% ---
\pdfbookmark[0]{\listtablename}{lot}
\listoftables*
\cleardoublepage
% ---
% inserir o sumario
% ---
\pdfbookmark[0]{\contentsname}{toc}
\tableofcontents*
\cleardoublepage
\textual

\chapter{Introduction}


\chapter{The model}\label{cap_trabalho_academico}

The model proposed by \citeonline{ho1981} assumes a market of homogeneous banks working as intermediary dealers. Each bank borrows money from clients to lend it to other clients. Borrowers and lenders arrive randomly on a Poisson distribution, the probability of a new deposit transaction and a new loan transaction are given by: 
\begin{align}
  \lambda_D = \alpha - \beta a \\
  \lambda_L = \alpha - \beta b
\end{align} 
Banks can influence the probability of a new deposit or loan by changing it's prices (making it more or less attractive for new clients). The prices for deposits $P_D$ and for loans $P_L$ are:
\begin{align}
  P_D = p+a \\
  P_L = p-b 
\end{align}
where $p$ is the "true" price of the loan or deposit, and $a$ and $b$ are fees that banks can use to increase the probability of a new deposit or loan.

The bank's wealth portfolio ($W$) is given by $W = Y + I + C$, where $Y$ is the base wealth which is invested in a diversified portfolio, $I$ is the credit inventory and $C$ is the money market position (or short-term net-cash) which is the difference between money market loans borrowings. The expected utility of wealth is given by the following equation:
\begin{equation}
  EU(W) = U(W_0) + U'(W_0)r_w W_0 + \frac{1}{2}U''(W_0)(\sigma_I^2 I_0^2 + 2\sigma_{IY} I_0 Y_0 + \sigma_Y^2 Y_0^2)
\end{equation}

The $r_w$ is the expected rate of return on wealth, $\sigma_I^2$ is the variance of the credit inventory, $\sigma_Y^2$ is the variance of the base wealth and $\sigma_{IY}$ is the covariance between $I$ and $Y$. For each new deposit transaction, the initial credit inventory changes by $Q$ and the credit inventory is $I_0 - Q$. The utility of one deposit transaction is:
\begin{equation}
  \begin{aligned}
    EU(W|\textit{one deposit transaction}) &= U(W_0) + U'(W_0)aQ +  U'(W_0)r_W W_0 \\ &+ \frac{1}{2}U''(W_0)(\sigma_I^2 Q^2 + 2\sigma_I^2 QI) \\ &+\frac{1}{2}U''(W_0)(\sigma_I^2 I_0^2 + 2\sigma_{IY} I_0 Y_0 + \sigma_Y^2 Y_0^2)
  \end{aligned}
\end{equation}

Similarly, for each new loan transaction, the credit inventory is $I_0 + Q$ and the utility of one loan transaction is:
\begin{equation}
  \begin{aligned}
    EU(W|\textit{one loan transaction}) &= U(W_0) + U'(W_0)bQ +  U'(W_0)r_W W_0 \\ &+ \frac{1}{2}U''(W_0)(\sigma_I^2 Q^2 - 2\sigma_I^2 QI) \\ &+\frac{1}{2}U''(W_0)(\sigma_I^2 I_0^2 + 2\sigma_{IY} I_0 Y_0 + \sigma_Y^2 Y_0^2)
  \end{aligned}
\end{equation}

\begin{equation}
  EU(W|a,b) = \lambda_D EU(W|\textit{one deposit transaction}) + \lambda_L EU(W|\textit{one loan transaction})
\end{equation}
Deriving the equation 2.8 with respect to the fee $a$ to maximize the wealth, we get to:
\begin{equation}
  \frac{\partial EU(W|a,b)}{\partial a} = -\beta \left( U'(W_0)aQ +\frac{1}{2} U''(W_0)\sigma_I^2 (Q^2 + QI) \right) + (\alpha - \beta a) U'(W_0)Q = 0
\end{equation}
From 2.9:
\begin{equation}
  a = \frac{1}{2} \frac{\alpha}{\beta} + \frac{1}{4} \left(-\frac{U''(W_0)}{U'(W_0)}\right) \sigma_I^2(Q + I)
\end{equation}
Doing the same for the fee $b$:
\begin{equation}
  \frac{\partial EU(W|a,b)}{\partial b} = -\beta \left( U'(W_0)bQ +\frac{1}{2} U''(W_0)\sigma_I^2 (Q^2 - QI) \right) + (\alpha - \beta b) U'(W_0)Q = 0
\end{equation}
\begin{equation}
  b = \frac{1}{2} \frac{\alpha}{\beta} + \frac{1}{4} \left(-\frac{U''(W_0)}{U'(W_0)}\right) \sigma_I^2(Q - I)
\end{equation}
The credit spread is defined by $S=a+b$. Substituting $a$ and $b$ above for $S$ we get:
\begin{equation}
  S = \frac{\alpha}{\beta} + \left(-\frac{U''(W_0)}{U'(W_0)}\right) \sigma_I^2Q
\end{equation}
Defining the Arrow-Pratt coefficient of absolute risk aversion as $R = -\frac{U''(W_0)}{U'(W_0)}$, we can rewrite the credit spread as:
\begin{equation}
  S = \frac{\alpha}{\beta} + R \sigma_I^2 Q
\end{equation}

This is the pure spread that will be estimated in the model, in which the first term $\alpha/\beta$ is the bank's neutral risk spread, which is the part of the risk that is affected by the competition. The second term is a first-order risk adjustment, which depends on the risk aversion ($R$), the size of the marginal transactions ($Q$) and the short-term variance of the interest rate ($\sigma_I^2$). The model's assumption of homogeneous banks implies that the terms in the credit spread are the same for all banks, even if the inventories ($I$) and the base wealth ($Y$) are different. 

The empirical model is built based on the theoretical model above but considering three market imperfections: (i) banks also pay implicit expenses through service fees and other costs indirectly involved in the transactions. (ii) the bank's opportunity cost of holding required reserves and (iii) the default risk on loans, for riskier borrowers, the bank may ask for a higher risk premium on the interest rate. The empirical model estimates the bank margins $M$ given by:
\begin{equation}
  M = f(S(\centerdot), IR, OR, DP, u)
\end{equation}

$S(\centerdot)$ is the pure spread defined in eqaution 2.14, $IR$ is the implicit interest expense, $OR$ is the opportunity cost of the required reserves, $DP$ is the default risk premiums on loans and $u$ is the error term.

\chapter{Sample and Data}

The data used in this study are mainly from the Central Bank of Brazil (Bacen) 
The variables used in this study for the bank $i$ in the period $t$ are:

\begin{table}[h!]
\centering
\begin{tabular}{ p{1.5cm} p{4cm} p{6cm} p{3cm}  }
\hline
Variable & Name & Description & Source \\ [0.5ex] 
\hline\hline
$D_{it}$ & Deposit interest rate & Rate paid on 30-day certificate of deposits & COSIF \\
$L_{it}$ & Loan interest rate & Average loan rate & COSIF \\
$S_{it}$ & Interest rate spread & $L_{it} - D_{it}$ & COSIF \\ 
[1ex] 
\hline
\end{tabular}
\caption{List of variables}
\label{table:1}
\end{table}

\chapter{Concluding Remarks}
% ---

\bibliographystyle{abntex2-alf}
\bibliography{bibliography}



\begin{apendicesenv}

\partapendices

\chapter{Expected utility of wealth}

The expected utility of wealth is given by the equation:
\begin{equation}
  E[U(W)] = U(W_0) + U'(W_0)r_w W_0 + \frac{1}{2}U''(W_0)(\sigma_I^2 I_0^2 + 2\sigma_{IY} I_0 Y_0 + \sigma_Y^2 Y_0^2)
\end{equation}

This function based on the assumption that the agents are risk averse, which is mathematically represented as a concave function, which is defined by the condition: $E[U(W)] < U(E[W])$. \citeonline{ingersoll1987} shows that 

\end{apendicesenv}
% ---


\begin{anexosenv}

% Imprime uma página indicando o início dos anexos
\partanexos

% ---
\chapter{Morbi ultrices rutrum lorem.}
% ---
\lipsum[30]

\end{anexosenv}

%---------------------------------------------------------------------
% INDICE REMISSIVO
%---------------------------------------------------------------------
\phantompart
\printindex
%---------------------------------------------------------------------

\end{document}
