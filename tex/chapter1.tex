\chapter{Introduction}
The Brazilian credit market has undergone significant changes in recent years, driven by new technologies and regulatory reforms that could potentially alter the competitive landscape of the banking sector. This study aims to understand some of the impacts of these changes on the interest rates charged by banks, particularly focusing on the interest rate spread. 

Brazilian banking industry is highly concentrated. On December 2024, the five largest banks hold 60.1\% of the market share on credit operations(for total loans), 52.8\% (for total deposits) and 45.3\% (for total assets) \footnote{Data from financial institutions accounting reports in \\ https://www.bcb.gov.br/estabilidadefinanceira/balancetesbalancospatrimoniais}. Even though the concentration is high, it doesn't mean that it is non-competitive, \citeonline{nakaneTestCompetitionBrazilian2001} tested the level of competitiveness and concluded that Brazilian market doesn't behave as a monopoly/cartel but institutions have some market power.

\citeonline{santosHighLendingInterest2021} used a Cournot model to estimate the effect of different factors on lending rates in Brazil between 2012 and 2016. He concludes that Brazilian rates are higher than other South American countries because of five main factors: IOF tax, high level of risk-free interest rate, higher probability of default, lower recovery rate and more concentrated financial system.
